\documentclass[a4paper]{article}
\usepackage{amsfonts,amssymb}
\input{style/ch_xelatex.tex}
\usepackage{geometry}
\usepackage{graphicx} 		%图片
\usepackage{subfigure}	
\geometry{a4paper, left=2.5cm, right=2.5cm, top=2.5cm, bottom=2.5cm}

\lstset{frame=, basicstyle={\footnotesize\ttfamily}}



\graphicspath{ {images/} }
\usepackage{ctex}
%-----------------------------------------BEGIN DOC----------------------------------------

\begin{document}
\renewcommand{\contentsname}{目\ 录}
\renewcommand{\appendixname}{附录}
\renewcommand{\appendixpagename}{附录}
\renewcommand{\refname}{参考文献} 
\renewcommand{\figurename}{图}
\renewcommand{\tablename}{表}
\renewcommand{\today}{\number\year 年 \number\month 月 \number\day 日}
\begin{figure}
	\centering
	\includegraphics[width=0.6\linewidth]{images/whu-logo}
\end{figure}
\title{{\Huge 数据科学基础课程论文{\large\linebreak\\}}{\Large 基于MNIST数据集的分类\linebreak\linebreak}}
\author{
    % 定义表格字体大小为 Large (大号)
    \Large
    % 增加表格行间距,使其看起来更舒展 (1.5倍行高)
    \renewcommand{\arraystretch}{1.5} 
    \begin{tabular}{r l}  % r表示第一列右对齐(冒号对齐),l表示第二列左对齐
        姓\qquad 名:& \;\;\;\,\,潘\ 宇\ 杰 \\
        学\qquad 号:& 2025286560200 \\
        专\qquad 业:& 25级应用统计
    \end{tabular}
    \vspace{8cm} 
    \\
    {\large 武汉大学}\\
    {\large 前沿交叉学科研究院}\\
    {\large 湖北国家应用数学中心}
}
% \author{\\姓\ 名:潘\ 宇\ 杰\\
% 学\ 号: 2025286560200\\
% 专\ 业: 应用统计\\\\\\\\\\\\\\\\\\\\\\\\\\
% 武汉大学\\
% 前沿交叉学科研究院\\
% 湖北国家应用数学中心\\
% }
\date{\today}
\maketitle
\newpage

%------------------------------------------TEXT--------------------------------------------

%----------------------------------------OVERVIEW-----------------------------------------
\section{数据集}

在计算机视觉与模式识别领域,MNIST数据库是最为经典且广泛使用的基准数据集之一。MNIST 是一个著名的手写数字分类任务基准,常被深度学习教材引用为入门实例。我们基于该数据集对卷积神经网络模型CNN进行了实现。

\subsection{数据规模与划分}

MNIST 数据集的构建旨在模拟真实的分类任务场景。该数据集包含总计 70,000 个样本,并被严格划分为两个子集:
\begin{itemize}
    \item \textbf{训练集}:包含 60,000 个样本,用于模型的参数学习。
    \item \textbf{测试集}:包含 10,000 个样本,用于评估模型。
\end{itemize}

\subsection{样本特征与预处理}

数据集中的每一个样本均为一张单通道的灰度图片,内容涵盖了从数字 0 到 9 的十个类别。
\begin{enumerate}
    \item \textbf{尺寸规范}:每个样本图像的物理尺寸被标准化为 $28 \times 28$ 像素。
    \item \textbf{位置规范}:为了降低特征提取的难度,MNIST 在预处理阶段对图像进行了中心化处理,确保每一个示例数字都位于图像的几何中心 。
\end{enumerate}
\begin{figure}[H]
    \centering
    \includegraphics[width=0.9\linewidth]{yangben.png}
    \caption{一张样本}
    \label{fig:placeholder}
\end{figure}
图 1 展示了示例数字及其对应的像素矩阵,直观地反映了每个像素点在矩阵中的亮度分布情况。

\newpage

\section{多分类问题误差分解}

\subsection{推导目标}

我们要证明期望超额风险满足:
\begin{equation}
\mathbb{E}_{\mathbb{D}}\left[\mathcal{E}\left(f_{\mathcal{A}(\tau)}\right)\right] \lesssim O(n^{-\text{rate}})
\end{equation}
并指明这个 rate(即老师所说的 $\alpha$)是如何计算出来的。
\subsection{第一步:误差分解 }
根据《FoDS\_25》 Page 5,对于训练集 $\mathbb{D}$ 大小为 $n$ 的学习算法,其超额风险可以分解为三部分:
\begin{equation}
\mathbb{E}[\mathcal{E}(\hat{f})] \le 2 \underbrace{\mathbb{E}[\sup_{f \in \mathcal{F}} |L(f) - \hat{L}(f)|]}_{\text{估计误差 (Estimation Error)}} + \underbrace{\inf_{f \in \mathcal{F}} (L(f) - L(f^*))}_{\text{近似误差 (Approximation Error)}} + \underbrace{\text{Opt}}_{\text{优化误差}}\end{equation}
\begin{itemize}
    \item 假设优化误差为 0(即假设我们能求得当前假设空间内的最优解)。
    \item 我们的目标是找到一个假设空间 $\mathcal{F}$(通常由分辨率参数 $\epsilon$ 控制),使得前两项之和最小。
\end{itemize}

\subsection{量化近似误差}

近似误差(即偏差)来源于模型假设空间 $\mathcal{F}$ 不足以完美表达真实的目标函数 $f^*$。根据《FoDS\_25》 Page 48 的设定,假设目标函数属于 Hölder 类 $H^\beta([0,1]^d)$,即函数的平滑度为 $\beta$(意味着函数通过 $\beta$ 阶导数有界),数据维度为 $d$ 。

如果我们使用网格大小(或分辨率)为 $\epsilon$ 的函数类来逼近 $f^*$,根据泰勒展开或多项式逼近理论,逼近误差的上界与分辨率的 $\beta$ 次方成正比:\begin{equation}\text{App Error} \le C_1 \cdot \epsilon^\beta\end{equation}
\begin{itemize}
    \item 物理含义:网格越密($\epsilon$ 越小),逼近越准,偏差越小。
\end{itemize}

\subsection{量化估计误差}

估计误差(即方差)来源于样本量有限。根据统计学习理论,估计误差的上界由假设空间的复杂度(覆盖数/熵)和样本量 $n$ 决定。

1、覆盖数:根据《FoDS\_25》 Page 48,对于平滑度为 $\beta$、维度为 $d$ 的函数类,其对数覆盖数为 :\begin{equation}\log \mathcal{N}(\epsilon) \asymp \epsilon^{-d/\beta}\end{equation}(注:文档中用 $\alpha$ 表示平滑度,此处为避免混淆,暂用 $\beta$ 表示平滑度,对应文档中的 $\alpha$)。

2、误差上界:对于多分类问题或有界损失函数,估计误差通常由 Dudley 积分或链式法则给出,其量级为:\begin{equation}\text{Est Error} \lesssim \sqrt{\frac{\log \mathcal{N}(\epsilon)}{n}} \asymp \sqrt{\frac{\epsilon^{-d/\beta}}{n}}\end{equation}(注:对于某些损失函数如平方损失,或者是满足 Tsybakov 噪声条件的分类,该项可能是 $\frac{\log \mathcal{N}}{n}$ 而不是根号,这会改变最终系数,但推导逻辑一致。这里采用最通用的根号形式)。
\begin{itemize}
    \item 物理含义:网格越密($\epsilon$ 越小),模型越复杂($\epsilon^{-d/\beta}$ 变大),方差越大;数据量 $n$ 越大,方差越小。
\end{itemize}

\subsection{偏差-方差权衡}

现在的总误差上界是 $\epsilon$ 的函数:\begin{equation}\text{Total Error}(\epsilon) \lesssim \underbrace{\epsilon^\beta}_{\text{Bias}} + \underbrace{\sqrt{\frac{\epsilon^{-d/\beta}}{n}}}_{\text{Variance}}\end{equation}为了得到最紧的上界(即最小的误差),我们需要选择最佳的 $\epsilon^*$,使这两项在数量级上平衡:\begin{equation}\epsilon^\beta \asymp \sqrt{\frac{\epsilon^{-d/\beta}}{n}}\end{equation}两边平方:\begin{equation}\epsilon^{2\beta} \asymp \frac{\epsilon^{-d/\beta}}{n}\end{equation}移项整理:\begin{equation}\epsilon^{2\beta} \cdot \epsilon^{d/\beta} \asymp \frac{1}{n}\end{equation}\begin{equation}\epsilon^{\frac{2\beta^2 + d}{\beta}} \asymp n^{-1}\end{equation}解出最佳分辨率 $\epsilon^*$:\begin{equation}\epsilon^* \asymp n^{-\frac{\beta}{2\beta + d}}\end{equation}
\subsection{得到最终误差上界}
将最佳 $\epsilon^*$ 代回总误差公式(由平衡条件可知,两项同阶,只需计算一项):\begin{equation}\text{Total Error} \asymp (\epsilon^*)^\beta \asymp \left( n^{-\frac{\beta}{2\beta + d}} \right)^\beta = n^{-\frac{\beta^2}{2\beta + d}}\end{equation}(注:如果根据文档 《FoDS\_25》  48 针对平方损失的推导,估计误差项是 $\frac{\epsilon^{-d/\beta}}{n}$ 而没有根号,那么平衡方程变为 $\epsilon^\beta \asymp \frac{\epsilon^{-d/\beta}}{n}$,解得 $\epsilon \asymp n^{-\frac{\beta}{2\beta+d}}$,最终误差为 $n^{-\frac{2\beta}{2\beta+d}}$。文档 《FoDS\_25》  48 明确给出的结果是 $O(n^{-\frac{2\alpha}{2\alpha+d}})$ 4)。无论具体是哪种损失函数的假设,最终形式都是幂律分布:
\begin{equation}
\mathbb{E}_{\mathbb{D}}\left[\mathcal{E}\right] \lesssim O(n^{-\text{rate}})
\end{equation}

\subsection{该证明过程依赖的假设}

1. 数据分布假设
\begin{itemize}
    \item 独立同分布 (i.i.d.):训练数据 $\mathcal{D} = \{Z_1, \dots, Z_n\}$ 必须是独立同分布的,采样自某个未知的联合分布 $\mu_Z$ 。这是所有统计学习界限(包括 Hoeffding 不等式、VC 维界限)生效的基础。
    \item 有界支撑集 :通常假设输入数据 $X$ 分布在一个紧致集合上,例如单位超立方体 $[0, 1]^d$ 。如果没有这个假设,数据的覆盖数可能无穷大,导致无法控制方差。
\end{itemize}
2. 目标函数的光滑性假设
这是决定指数 $\alpha$ 大小的最关键假设。
\begin{itemize}
    \item Hölder 连续性 :假设真实的贝叶斯最优分类器 $f^*$(或回归函数)属于 Hölder 函数类 $\mathcal{H}^\beta([0, 1]^d)$。
    \begin{itemize}
    \item 这意味着目标函数 $f^*$ 具有 $\beta$ 阶的平滑度(即函数不会剧烈震荡,导数有界)。
    \item 作用:这个假设限制了近似误差。如果函数不光滑(任意乱跳),无论模型多复杂都无法用有限网格逼近,偏差将无法收敛。
    \item 推导中的 $\alpha$ 通常与平滑度 $\beta$ 直接相关(例如 $\alpha = \frac{\beta}{2\beta+d}$)。
    \end{itemize}
\end{itemize}
3. 假设空间的复杂度假设
\begin{itemize}
    \item 有限的覆盖数
    :假设模型函数类 $\mathcal{F}$ 的复杂度是可以被度量的。具体来说,其对数覆盖数 $\log \mathcal{N}(\xi)$ 必须随着分辨率 $\xi$ 的提高以特定速率增长,通常假设为 $\log \mathcal{N}(\xi) \asymp \xi^{-d/\beta}$ 。
    \begin{itemize}
    \item 作用:这个假设用于控制估计误差。它保证了随着样本量增加,我们能以足够快的速度“填满”整个空间,从而找到最优解。
    \end{itemize}
\end{itemize}
4. 优化与模型选择假设
\begin{itemize}
    \item 偏差-方差的最优权衡:证明过程假设我们能够选取最优的模型复杂度(例如最优的网格大小 $\xi^*$ 或神经网络的宽度/深度)。
    \begin{itemize}
    \item 文档第 48 页展示了如何选取 $\xi^* = n^{-\frac{\alpha}{d+2\alpha}}$ 来平衡 Bias 和 Variance。
    \item 隐含假设:如果模型复杂度(如网络层数)固定不变,或者优化算法未能找到最优解(Optimization Error $\neq 0$),则该收敛速率 $O(n^{-\alpha})$ 无法达到。
    \end{itemize}
\end{itemize}
5. 噪声/边界条件假设 —— 针对分类问题虽然《FoDS\_25》 第 48 页主要以回归(最小二乘)为例,但在分类问题中要达到 $O(n^{-\alpha})$ 这种形式的快速率,通常还需要:
\begin{itemize}
    \item Tsybakov 噪声条件:假设在决策边界附近,类别概率 $P(Y=1|X)$ 既然不等于 0.5,也不会无限逼近 0.5(或者逼近的速度受控)。
    \begin{itemize}
    \item 《FoDS\_25》 第 74 页提到了 $\sigma_s, \sigma_t$ 等参数以及与之相关的误差界,这通常对应于迁移学习或分类任务中的边界条件假设。如果没有这个假设,分类问题的误差界通常只能达到 $O(n^{-1/2})$ 甚至更慢,而无法达到与平滑度 $\alpha$ 相关的更快得多的速率。
    \end{itemize}
\end{itemize}

\newpage

\section{网络架构设计}

\subsection{整体概述}

模型输入数据为 $1 \times 28 \times 28$ 的灰度图像,经过两个阶段的卷积特征提取模块后,数据被映射为高维语义特征,最后通过全连接层完成 10 分类的概率输出。整个网络包含 4 个卷积层和 2 个全连接层,并在层间引入了批归一化和 Dropout 以防止过拟合。

\subsection{特征提取}

特征提取部分由2个结构相似但深度递增的卷积块组成。

\subsubsection{第一阶段:浅层特征提取}
第一阶段旨在从原始像素中提取边缘、纹理等基础几何特征。
该模块包含两个连续的卷积层:
\begin{itemize}
    \item \textbf{卷积$\mathtt{Conv2d(1, 32, 3, padding=1)}$}:使用 $3 \times 3$ 的卷积核,并采用padding=1填充,确保卷积操作不改变特征图的空间尺寸。
    \item \textbf{归一化与激活$\mathtt{nn.BatchNorm2d(32),nn.ReLU()}$}:每个卷积层后紧接 BN 层和 ReLU 激活函数。公式如下:
    \begin{equation}
        Y = \text{ReLU}(\text{BN}(W * X + b))
    \end{equation}
    经过\textbf{双重卷积}后,特征图维度变为 $32 \times 28 \times 28$。
    \item \textbf{池化层$\mathtt{nn.MaxPool2d(2)}$}:随后,网络应用一个 $2 \times 2$ 的最大池化层,步长为 2。这一步执行下采样操作,将特征图的空间分辨率减半,从而减少计算量并引入平移不变性。
    \item \textbf{$\mathtt{nn.Dropout(0.25)}$}:该模块引入丢弃率为 0.25 的 Dropout 层。第一阶段的最终输出张量维度为 $32 \times 14 \times 14$。
\end{itemize}

\subsubsection{第二阶段:深层特征提取}
第二阶段同样包含两个连续的卷积层,但输出通道数提升至 64。通过增加通道数,网络能够在低分辨率的空间网格上编码更丰富的信息量。
\begin{itemize}
    \item \textbf{卷积$\mathtt{nn.Conv2d(32, 64, 3, padding=1)}$}:输入 $32 \times 14 \times 14$ $\rightarrow$ 两个 $64 \times 3 \times 3$ 卷积层 $\rightarrow$ 特征图 $64 \times 14 \times 14$。
    \item \textbf{归一化与激活$\mathtt{nn.BatchNorm2d(32),nn.ReLU()}$}:每个卷积层后紧接 BN 层和 ReLU 激活函数。
    \item \textbf{池化层$\mathtt{nn.MaxPool2d(2)}$}:再次经过 $2 \times 2$ 最大池化层,空间尺寸进一步缩小至 $7 \times 7$。
    \item \textbf{$\mathtt{nn.Dropout(0.25)}$}:最后,该模块引入丢弃率为 0.25 的 Dropout 层。
\end{itemize}
经过本阶段处理,原始图像被编码为 $64 \times 7 \times 7$ 的高维特征张量。相较于原始输入,该特征张量虽然空间分辨率较低,但每个“像素”点都包含了原始图像中较大感受野。
\begin{figure}[H]
    \centering
    \includegraphics[width=0.9\linewidth]{wangluojiagou.png}
    \caption{网络架构图}
    \label{fig:wangluojiagou}
\end{figure}

\subsection{分类器模块}

网络的末端是一个MLP,负责将提取的三维特征映射到类别标签空间。
\begin{enumerate}
    \item \textbf{展平}:首先,将第二阶段输出的 $64 \times 7 \times 7$ 张量展平为一维向量,长度为 $N = 64 \times 7 \times 7 = 3136$。
    \item \textbf{全连接层$\mathtt{nn.Linear(64 * 7 * 7, 128)}$}:第一层线性变换将 3136 维向量映射到 128 维的隐藏空间。该层同样配备了BN和 ReLU 激活函数,以加速收敛并防止梯度消失。
    \item \textbf{强正则化$\mathtt{nn.Dropout(0.5)}$}:为了显著降低全连接层的过拟合风险,此处引入了丢弃率为 0.5 的 Dropout 层。每次迭代将随机丢弃一半的神经元,迫使网络学习更加鲁棒的分布式特征表示。
    \item \textbf{输出层$\mathtt{nn.Linear(128, 10)}$}:最后一个全连接层将 128 维特征映射到 10 维输出,分别对应 MNIST 数据集中的数字 0 至 9。
\end{enumerate}
最终,输出向量通过 Softmax 函数转化为预测概率分布。

\subsection{参数与结构汇总}
表 \ref{tab:arch} 总结了各层的详细参数配置。
\begin{table}[H]
    \centering
    \caption{神经网络架构详细参数表}
    \label{tab:arch}
    \begin{tabular}{lccccc}
        \hline
        \textbf{块名称} & \textbf{层级名称} & \textbf{输入尺寸} & \textbf{核大小/步长} & \textbf{输出通道} & \textbf{输出尺寸} \\
        \hline
        \multirow{3}*{\textbf{Block 1}} & Conv2d $\times 2$ & $1 \times 28 \times 28$ & $3 \times 3$ / 1 & 32 & $28 \times 28$ \\
         & Max Pool & $32 \times 28 \times 28$ & $2 \times 2$ / 2 & 32 & $14 \times 14$ \\
         & Dropout (0.25) & - & - & - & - \\
        \hline
        \multirow{3}*{\textbf{Block 2}} & Conv2d $\times 2$ & $32 \times 14 \times 14$ & $3 \times 3$ / 1 & 64 & $14 \times 14$ \\
         & Max Pool & $64 \times 14 \times 14$ & $2 \times 2$ / 2 & 64 & $7 \times 7$ \\
         & Dropout (0.25) & - & - & - & - \\
        \hline
        \multirow{4}*{\textbf{Classifier}} & Flatten & $64 \times 7 \times 7$ & - & - & 3136 \\
         & Linear & 3136 & - & 128 & 128 \\
         & Dropout (0.5) & - & - & - & - \\
         & Output & 128 & - & 10 & 10 \\
        \hline
    \end{tabular}
\end{table}

\subsection{损失函数与经验风险最小化}

\subsubsection{交叉熵损失函数}

在前述的理论推导中,我们讨论了在假设空间 $\mathcal{F}$ 中寻找最优函数 $f^*$ 以最小化期望风险。在具体的分类任务实现中,为了衡量模型输出分布与真实标签分布之间的差异,本项目选用了交叉熵损失函数。

设输入样本为 $x$,真实标签为 $y$(采用 One-hot 编码,即 $y \in \{0,1\}^K$,其中 $K=10$),模型输出的未归一化 Logits 为 $z = f_\theta(x)$。首先通过 Softmax 函数将输出映射为概率分布 $\hat{y}$:
\begin{equation}
    \hat{y}_k = \text{Softmax}(z)_k = \frac{e^{z_k}}{\sum_{j=1}^K e^{z_j}}
\end{equation}
在此基础上,单样本的交叉熵损失定义为:
\begin{equation}
    L(\hat{y}, y) = - \sum_{k=1}^K y_k \log(\hat{y}_k)
\end{equation}
由于 MNIST 数据集的标签是稀疏的(仅目标类别 $c$ 处 $y_c=1$,其余为 0),该损失函数简化为负对数似然:
\begin{equation}
    L(\hat{y}, y) = - \log(\hat{y}_c)
\end{equation}
相较于均方误差(MSE),交叉熵损失函数在分类问题中具有更优的梯度特性。当模型预测错误时,$-\log(\hat{y}_c)$ 的梯度较大,能够加速收敛;同时,它在概率意义上等价于最大似然估计(MLE),符合统计学习的一致性。

\subsubsection{经验风险最小化 (ERM) 的实现}

根据第 2 节的推导,我们的目标是最小化期望风险 $R(f) = \mathbb{E}[L(f(X), Y)]$。然而,由于联合分布 $\mathcal{D}$ 未知,在实际训练中,我们通过最小化训练集 $\mathcal{S} = \{(x_i, y_i)\}_{i=1}^n$ 上的经验风险 $\hat{R}_n(f)$ 来逼近这一目标:
\begin{equation}
    \min_{\theta} \hat{R}_n(f_\theta) = \min_{\theta} \frac{1}{n} \sum_{i=1}^n L(f_\theta(x_i), y_i)
\end{equation}
在代码实现中,这一过程通过以下方式体现:
\begin{itemize}
    \item \textbf{前向传播}:计算 Mini-batch 内样本的平均交叉熵损失,作为经验风险 $\hat{R}_n$ 的无偏估计。
    \item \textbf{反向传播}:利用链式法则计算 $\nabla_\theta \hat{R}_n$,并使用 Adam 优化器更新参数 $\theta$。
\end{itemize}
尽管第 2.2 节理论推导中假设优化误差为 0,但神经网络的损失曲面是非凸的。为了防止模型陷入局部极小值并控制泛化误差中的“估计误差”项,我们在实现中引入了 \texttt{Dropout} 正则化和学习率衰减策略,以此在降低经验风险的同时,约束假设空间的复杂度,防止过拟合。

\subsection{损失函数的概率解释与梯度推导}

\subsubsection{最大似然估计与交叉熵的等价性}

在统计学习框架下,我们的神经网络 $f_\theta$ 实际上在拟合条件概率分布 $P(Y|X)$。假设训练样本独立同分布(i.i.d.),对于数据集 $\mathcal{D} = \{(x_i, y_i)\}_{i=1}^n$,参数 $\theta$ 的最大似然估计(MLE)目标为最大化对数似然函数:
\begin{equation}
    \theta_{MLE} = \arg\max_{\theta} \sum_{i=1}^n \log P(y_i | x_i; \theta)
\end{equation}
在多分类任务中,模型输出的概率分布由 Softmax 函数给出,即 $P(Y=k|x) = \hat{y}_k$。对于单样本 $(x, y)$,其中 $y$ 为 One-hot 标签向量,似然项为 $\prod_{k=1}^K \hat{y}_k^{y_k}$。取负对数后,最小化负对数似然等价于最小化交叉熵损失:
\begin{equation}
    -\log \mathcal{L}(\theta) = -\sum_{i=1}^n \sum_{k=1}^K y_{i,k} \log(\hat{y}_{i,k}) = \sum_{i=1}^n L(\hat{y}_i, y_i)
\end{equation}
这从概率论角度严格证明了为何在 ERM 框架下选择交叉熵作为损失函数。

\subsubsection{Softmax-CrossEntropy 的梯度解析推导}
为了深入理解模型,我们推导损失函数 $L$ 关于神经网络输出层 Logits $z$ 的梯度。这是反向传播算法起点的核心推导。

设 logits 向量为 $z = (z_1, \dots, z_K)^T$,经过 Softmax 得到 $\hat{y}_k = \frac{e^{z_k}}{\sum_{j} e^{z_j}}$。损失函数为 $L = -\sum_{k} y_k \log \hat{y}_k$。
利用链式法则,我们计算分量 $z_i$ 的偏导数 $\frac{\partial L}{\partial z_i}$:
\begin{equation}
    \frac{\partial L}{\partial z_i} = -\sum_{k=1}^K y_k \frac{\partial \log \hat{y}_k}{\partial z_i} = -\sum_{k=1}^K \frac{y_k}{\hat{y}_k} \frac{\partial \hat{y}_k}{\partial z_i}
\end{equation}
其中 Softmax 函数的雅可比矩阵元素 $\frac{\partial \hat{y}_k}{\partial z_i}$ 具有如下性质:
\begin{equation}
    \frac{\partial \hat{y}_k}{\partial z_i} = 
    \begin{cases} 
    \hat{y}_i (1 - \hat{y}_i) & \text{if } k = i \\
    -\hat{y}_k \hat{y}_i & \text{if } k \neq i 
    \end{cases}
\end{equation}
将其代入梯度公式:
\begin{equation}
\begin{aligned}
    \frac{\partial L}{\partial z_i} &= - \frac{y_i}{\hat{y}_i} \hat{y}_i(1-\hat{y}_i) - \sum_{k \neq i} \frac{y_k}{\hat{y}_k} (-\hat{y}_k \hat{y}_i) \\
    &= -y_i(1-\hat{y}_i) + \sum_{k \neq i} y_k \hat{y}_i \\
    &= -y_i + y_i \hat{y}_i + \hat{y}_i \sum_{k \neq i} y_k \\
    &= -y_i + \hat{y}_i \left( y_i + \sum_{k \neq i} y_k \right)
\end{aligned}
\end{equation}
由于 $y$ 是 One-hot 向量,$\sum_{k} y_k = 1$,因此括号内各项之和为 1。最终我们得到一个非常优雅的结论:
\begin{equation}
    \frac{\partial L}{\partial z_i} = \hat{y}_i - y_i
\end{equation}
\textbf{物理意义}:损失函数关于 Logits 的梯度恰好是\textbf{预测概率与真实标签的差值}。这一线性形式保证了当预测误差较大时,梯度范数也较大,从而使参数更新具有很强的纠错能力,避免了使用均方误差(MSE)配合 Sigmoid/Softmax 时容易出现的梯度消失问题。这正是本项目采用该架构能实现 $O(n^{-rate})$ 收敛速率的关键数值基础。
\newpage

\section{实验设置与结果分析}

\subsection{模型实现与训练策略}

\subsubsection{数据增强}

本实验在数据预处理阶段引入了数据增强。具体设置如下:
\begin{itemize}
    \item \textbf{随机旋转}:在 $\pm 15^{\circ}$ 范围内对图像进行随机旋转。
    \item \textbf{随机仿射变换}:在水平和垂直方向进行 $10\%$ 范围内的随机平移。
\end{itemize}
这种策略迫使模型主动学习而非只固定校对像素位置,显著提升了模型的鲁棒性。

\subsubsection{训练超参数配置}

实验采用 \textbf{Adam} 优化器,利用其自适应矩估计能力加快收敛速度。初始学习率设为 $1 \times 10^{-3}$,批次大小为 50。
本实验还采用了 \texttt{StepLR} 学习率调度策略:每经过 2 个 Epoch,学习率衰减为当前的 $0.1$ 倍。这种策略使得模型在初期快速下降 Loss,在后期微调参数以逼近全局最优解。

\subsection{实验结果分析}

实验在 Kaggle平台上的免费GPU T4×2 下进行,共训练 7 个 Epoch,测试集包含 2,000 个未参与训练的独立样本。

\subsubsection{精度表现}

模型在第 6 个 Epoch 达到了历史最佳测试精度 \textbf{99.50\%},超过了 99\% 的目标。

\subsubsection{收敛性分析}

得益于 Batch Normalization 和 Adam 优化器的结合,模型展现了极快的收敛特性:
\begin{itemize}
    \item 仅在 Epoch 1 后,测试精度即达到 $98.75\%$。
    \item 在 Epoch 2 结束时,精度迅速突破 $99.0\%$ 关口,达到 $99.10\%$。
\end{itemize}
此外,观察训练日志可以发现,在 Epoch 2 和 Epoch 4 的学习率衰减节点之后,模型的 Loss 波动明显减小,精度稳步提升并稳定在 $99.2\%$ 以上。最终 Epoch 的平均精度保持在 $99.25\%$ 左右,表明模型已经收敛至一个鲁棒的极值点,未出现明显的过拟合现象。
\begin{figure}[H]
	\centering
	\subfigure {\includegraphics[width=.45\textwidth]{xunlian2.png}}
	\subfigure {\includegraphics[width=.45\textwidth]{xunlian.png}}
	\caption{训练初期与后期结果}
	\label{fig_E1}
\end{figure}

\section{代码存放网址}

https://github.com/Mercury481/FoDS\_pyj.git

\begin{thebibliography}{9}
\bibitem{1} JIAO Yuling. Foundation of Data Science [EB/OL]. 2025 [2025-12-26].
\bibitem{2} LIU Shicai. minist [EB/OL]. 2017 [2025-12-26]. https://www.kaggle.com/code/lsc2data/minist.
\end{thebibliography}

\end{document}




